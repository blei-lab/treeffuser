\documentclass{article}


% if you need to pass options to natbib, use, e.g.:
%     \PassOptionsToPackage{numbers, compress}{natbib}
% before loading neurips_2023


% ready for submission
\usepackage{neurips_2023}


% to compile a preprint version, e.g., for submission to arXiv, add add the
% [preprint] option:
%     \usepackage[preprint]{neurips_2023}


% to compile a camera-ready version, add the [final] option, e.g.:
%     \usepackage[final]{neurips_2023}


% to avoid loading the natbib package, add option nonatbib:
%    \usepackage[nonatbib]{neurips_2023}


\usepackage[utf8]{inputenc} % allow utf-8 input
\usepackage[T1]{fontenc}    % use 8-bit T1 fonts
\usepackage{hyperref}       % hyperlinks
\usepackage{url}            % simple URL typesetting
\usepackage{booktabs}       % professional-quality tables
\usepackage{amsfonts}       % blackboard math symbols
\usepackage{nicefrac}       % compact symbols for 1/2, etc.
\usepackage{microtype}      % microtypography
\usepackage{xcolor}         % colors

%include math packages
\usepackage{amsmath}
\usepackage{amssymb}
\usepackage{amsthm}
\usepackage{mathtools}
\usepackage{bm}
\usepackage{bbm}
\usepackage{dsfont}
\usepackage{mathrsfs}
\usepackage{algorithm}

% Define a command for bold symbols
\newcommand{\bs}[1]{\boldsymbol{#1}}

% Probabilistic notation
\newcommand{\EE}{\mathbb{E}}
\newcommand{\Var}{\text{Var}}
\newcommand{\Cov}{\text{Cov}}
\newcommand{\KL}{\text{KL}}
\newcommand{\Prob}{\mathbb{P}}


\title{Formatting Instructions For NeurIPS 2023}


% The \author macro works with any number of authors. There are two commands
% used to separate the names and addresses of multiple authors: \And and \AND.
%
% Using \And between authors leaves it to LaTeX to determine where to break the
% lines. Using \AND forces a line break at that point. So, if LaTeX puts 3 of 4
% authors names on the first line, and the last on the second line, try using
% \AND instead of \And before the third author name.


\author{%
  David S.~Hippocampus\thanks{Use footnote for providing further information
    about author (webpage, alternative address)---\emph{not} for acknowledging
    funding agencies.} \\
  Department of Computer Science\\
  Cranberry-Lemon University\\
  Pittsburgh, PA 15213 \\
  \texttt{hippo@cs.cranberry-lemon.edu} \\
  % examples of more authors
  % \And
  % Coauthor \\
  % Affiliation \\
  % Address \\
  % \texttt{email} \\
  % \AND
  % Coauthor \\
  % Affiliation \\
  % Address \\
  % \texttt{email} \\
  % \And
  % Coauthor \\
  % Affiliation \\
  % Address \\
  % \texttt{email} \\
  % \And
  % Coauthor \\
  % Affiliation \\
  % Address \\
  % \texttt{email} \\
}


\begin{document}


\maketitle


\begin{abstract}
  The abstract paragraph should be indented \nicefrac{1}{2}~inch (3~picas) on
  both the left- and right-hand margins. Use 10~point type, with a vertical
  spacing (leading) of 11~points.  The word \textbf{Abstract} must be centered,
  bold, and in point size 12. Two line spaces precede the abstract. The abstract
  must be limited to one paragraph.
\end{abstract}


\section{Submission of papers to NeurIPS 2023}

% include ./section/introduction.tex
\section{Goals}

Assume that $\mathbf{x} \in \mathbb{R}^{d_x}$, $\mathbf{y} \in \mathbb{R}^{d_y}$
and that there exists some distribution $p^*(\mathbf{x}, \mathbf{y})$ over both $\mathbf{x}$ and $\mathbf{y}$.
The goal is to learn the full conditional distribution $p^*(\mathbf{y}|\mathbf{x})$ so that we can sample from it.

Once we have learned to sample from it we can use the samples as one would in a bayesian setting and report, mean,
quantiles, variance, expectations of quantities, measures of risk and uncertainty (e.g CVAR, VaR, etc.), etc.

The requirements for this algorithm are:
\begin{enumerate}
    \item It should be able to learn the full conditional distribution $p^*(\mathbf{y}|\mathbf{x})$
    even if the distribution is multimodal, has complex dependencies, is reasonably high-dimensional,
    has heavy tails, etc.
    \item It should be robust enough to be usable by non-experts with a scikit-learn like interface.
    \item The algorithm should be tailored for tabular data (i.e not for images, text, etc.).
\end{enumerate}


\section{The algorithm}
The idea of the algorithm is the following:
% add vertical space
\vskip 0.1in

\textit{Use diffusion models to learn the conditional distribution $p^*(\mathbf{y}|\mathbf{x})$
but instead of estimating the score function using a neural network use gradient
boosted trees, with one tree for each dimension of $\mathbf{y}$.}
\vskip 0.1in


To be concrete, we first construct a diffusion process $\{ \mathbf{y}(t)\}_{t=0}^T$,
indexed by time $t \in [0, T]$, with initial condition $\mathbf{y}(0) \sim p^*(\mathbf{y}|\mathbf{x})$
and such that
\[ d \mathbf{y} = \mathbf{f}(\mathbf{y}, t) dt + g(t) d \mathbf{W} \]
where $\mathbf{W}$ is a Wiener process and $\mathbf{f}$ and $g$ are functions that we
choose.
\vskip 0.1in

By a result in Anderson (I'll add the reference) then it is the case that by starting from samples
$\mathbf{y}(T) \sim p^*(\mathbf{y}(T)|\mathbf{x})$ and reverting the process
we can obtain samples from $p^*(\mathbf{y}|\mathbf{x})$.
Moreover, the reverse process has the nice form of
\begin{align*}
\label{eq:reverse_process}
d \mathbf{y} = [\mathbf{f}(\mathbf{y}, t) - g(t)^2 \nabla_{\mathbf{y}} \log p_t(\mathbf{y}| \mathbf{x})] dt +  g(t) d \mathbf{\bar{W}}
\end{align*}
where $p_t(\mathbf{y}| \mathbf{x})$ is the density of the diffusion process at time $t$, and $\mathbf{\bar{W}}$ is a reverse Wiener process.
\vskip 0.1in

Therefore if we can learn the score function $\nabla_{\mathbf{y}} \log p_t(\mathbf{y}| \mathbf{x})$
and we construct the diffusion process such that $p^*(\mathbf{y}(T)|\mathbf{x})$ is a
simple distribution like a standard normal then we can sample from $p^*(\mathbf{y}|\mathbf{x})$
by sampling from the standard normal and then running the reverse process
(this can be done with standard numerical methods for SDEs). The only thing
that we need is an estimate for the score function.
\vskip 0.1in

It turns out that by results in \cite{vincent2010connection} it is possible to estimate this score function by
 selecting the mapping $\mathbf{s}(\mathbf{y}, \mathbf{x}, t): \mathbb{R}^{d_y} \times \mathbb{R}^{d_x} \times \mathbb{R} \to \mathbb{R}^{d_y}$
 that minimizes the following loss:

\begin{align}
\mathcal{L}(\mathbf{s}) = \E_{p(t)}\E_{p^*(\mathbf{x})p^*(\mathbf{y}|\mathbf{x}) p_t(\mathbf{y}(t)|\mathbf{y}(0))} \left[ \norm{ \nabla_{\mathbf{y}} \log p_{t}(\mathbf{y}(t)|\mathbf{y}(0)) - \mathbf{s}(\tilde{\mathbf{y}}, \mathbf{x}, t) }^2 \right]
\end{align}
Where $p(t)$ is uniform over $[0, T]$, and $p_t(\mathbf{y}(t)|\mathbf{y}(0))$ is the density of the diffusion process at time $t$ conditioned
on the initial condition $\mathbf{y}(0)$.

This quantity is something that we can work with empirically because
\begin{enumerate}
    \item $ \nabla_{\mathbf{y}} \log p_{t}(\mathbf{y}(t)|\mathbf{y}(0))$ is computable analytically as we construct the
    diffusion process.
    \item We can sample from $p(t)$ because it is just a uniform distribution.
    \item We can motecarlo approximate the expectation over $p^*(\mathbf{x})p^*(\mathbf{y}|\mathbf{x}) p_t(\mathbf{y}(t)|\mathbf{y}(0))$.
    by using a dataset $\{ \mathbf{x}_i, \mathbf{y}_i \}_{i=1}^n$ and sampling from $p(\mathbf{y}(t)|\mathbf{y}(0))$
    which is ok because it is a quantity we construct ourselves when we create the diffusion process.
\end{enumerate}

Therefore, what we can do is construct one gradient boosted tree for each dimension of $\mathbf{y}$
that takes as input $\mathbf{y}$ and $\mathbf{x}$ and $t$  outputs the i-th coordinate of the score function
$s(\mathbf{y}, \mathbf{x}, t)_i$ and we train them by minimizing the loss above.

Once trained we have a function $\mathbf{s}(\mathbf{y}, \mathbf{x}, t)$ that we can plug-in to the reverse SDE
(\ref{eq:reverse_process}) and use to sample from $p^*(\mathbf{y}|\mathbf{x})$ which is what we wanted.


\section{Why Gradient Boosted Trees?}
In principle this scheme could work with any model.
However, gradient boosted trees are the method of choice for tabular datasets.
Moreover, because we care about robustness and usability by non-experts
they seem like the natural model class.

In some simple experiments that we have conducted in a dataset with $n=10000$ and $d_x=1$ and $d_y=1$
it takes around 3 seconds to train and 20 seconds to sample $100$ points for a $1000$ test points
(i.e $100000$ samples total).

\section{Why better calibration - UQ?}
We make no parametric assumptions about the form of the likelihood
and hence we can learn the conditional distribution $p^*(\mathbf{y}|\mathbf{x})$
regardless of how complex it is as long as we sample densely enough from it and our
model is flexible enough.(We could even prove stuff about this but not sure it is helpful).

\section{Diffusion Models}
In this section we quckly review the basics of diffusion models.
We focus on the stochastic differential equation formulation
first presented by \citet{song2021scorebased}.

Let $p(\bs{y})_\text{data}$ denote the data distribution.
The goal of a diffusion model is to learn a mapping from a
simple distribution $p(\bs{z})$ to the data distribution $p(\bs{y})_\text{data}$.

This is achived by reversing a diffusion process.
In particular, we construct a stochastic differential equation $\bs{y}(t)$ from
$t \in [0, T]$
such that $\bs{y}(0) \sim p(\bs{y})_\text{data}$ and $\bs{y}(1) \sim p(\bs{y}(T))$
is a simple distribution we can sample from and whose evolution is given by
\begin{align}
  d \bs{y}(t) = \bs{f}(\bs{y}(t), t) dt + \bs{g}(t) d\bs{w}(t),
\end{align}
where $\bs{w}(t)$ is a standard Brownian motion and $\bs{f} : \mathbb{R}^d \times [0, T] \to \mathbb{R}^d$
and $\bs{g} : [0, T] \to \mathbb{R}$ are called the drift coefficient and the diffusion coefficient, respectively.

It is possible to reverse this SDE and sample from $p(\bs{y})_\text{data}$ by first sampling
$\bs{y}(1) \sim p(\bs{y}(T))$ and then evolving the system backwards in time. This is done by solving the
reverse SDE (Cite anderson 1982)

\begin{align}
    d \bs{y}(t) = [f(\bs{y}(t), t) - g(t)^2 \nabla_{\bs{y}(t)} \log p(\bs{y}(t))] dt + g(t) d\bar{\bs{w}}(t).
  \end{align}
where $\bar{\bs{w}}(t)$ is a standard Brownian with reversed time. Thus because $\bs{f}$ and $g$ are known,
and we construct the SDE so that $p(\bs{y}(T))$ is simple, as long
as we know the score $\nabla_{\bs{y}(t)} \log p(\bs{y}(t))$ we can sample from $p(\bs{y})_\text{data}$.

\subsection*{Estimating the Score}
An important result by \citet{Vincent2010} is that it is possible to estimate the score
$\nabla_{\bs{y}(t)} \log p(\bs{y}(t))$ by computing
\begin{align}
    \bs{s}^* = \text{argmin}_{\bs{s} \in \mathcal{S}}  \EE_{t}\EE_{p(\bs{y}(0))_\text{data}}\EE_{p(\bs{y}(t)|\bs{y}(0))} \left[ \left\| \nabla_{\bs{y}(t)} \log p(\bs{y}(t)| \bs{y}(0)) - \bs{s}(\bs{y}(t), t) \right\|^2 \right].
\end{align}
where $\mathcal{S} = \{ \bs{s}: \mathbb{R}^d \times [0, T] \to \mathbb{R}^d \}$ is the set of all possible score functions
indexed by time $t$, and $\EE_{t}$ denotes the expectaion over uniformly sampled $t \in [0, T]$.

\subsection*{Conditional Diffusion Models}
Although it is most common to train diffusion models unconditionally as explained above, one can also train
diffusion models conditionally on some input $\bs{x}$.

To do so we make the following modifications to the above formulation.
\begin{enumerate}
    \item We construct one separate SDE per value of $\bs{x}$. Each SDE shares the same drift and diffusion coefficients
    but the initial distribution $p_{\bs{x}}(\bs{y}(0))$ is given by $p(\bs{y}| \bs{x})_{\text{data}}$.
    \item The reverse SDE is now given by
    \begin{align}
        d \bs{y}(t) = [f(\bs{y}(t), t) - g(t)^2 \nabla_{\bs{y}(t)} \log p_{\bs{x}}(\bs{y}(t))] dt + g(t) d\bar{\bs{w}}(t).
    \end{align}
    where $\nabla_{\bs{y}(t)} \log p_{\bs{x}}(\bs{y}(t))$ is the score of the conditional distribution $p_{\bs{x}}(\bs{y}(t))$.
    Importantly, because we choose the diffusion and drift coefficients so that at $t = T$ the distribution is the same
    for all values of $\bs{x}$, we can still sample from the data distribution in the same way as before.
    \item The final change is that the score function is now estimated by
    \begin{align*}
    \bs{s}^* = \text{argmin}_{\bs{s} \in \mathcal{S}}  \EE_{t}\EE_{p(\bs{y}(0), \bs{x})_\text{data}}\EE_{p(\bs{y}(t)|\bs{y}(0))} \left[ \left\| \nabla_{\bs{y}(t)} \log p(\bs{y}(t)| \bs{y}(0)) - \bs{s}(\bs{y}(t),\bs{x}, t) \right\|^2 \right].
    \end{align*}
    with the changes being that now $\mathcal{S} = \{ \bs{s}: \mathbb{R}^d \times \mathbb{R}^m \times [0, T] \to \mathbb{R}^d \}$ is the set of all possible
    score functions but now allowing for the score to depend on the input $\bs{x}$, and the expectation is taken over the joint distribution
    $p(\bs{y}(0), \bs{x})_\text{data}$.
    We emphasize that the score of the conditional distribution $p(\bs{y}(t)| \bs{y}(0))$ is still
    the same because once we condition on $\bs{y}(0)$ the distribution is the same for all values of $\bs{x}$.

\end{enumerate}
This formulation of conditional diffusion models is different than controllable generation as
presented in \cite{song2021scorebased}. There, a conditional diffusion model is constructed by noting
that
\[ \nabla_{\bs{y}(t)} p(\bs{y}(t)| \bs{x}) = \nabla_{\bs{y}(t)} p(\bs{y}(t)) + \nabla_{\bs{y}(t)} p(\bs{x}| \bs{y}(t)) \]
and hence if we obtain the first term from an unconditional diffusion model, and the second term by differentiating
through another trained model $p(\bs{x}| \bs{y}(t))$, we can obtain the score of the conditional distribution.
In our case this is not feasible because in general the dimension of $\bs{y}$ will be much smaller than the dimension of $\bs{x}$.


\section{Gradient Boosted Trees}
Gradient Boosted Trees (GBT) \cite{friedman2001greedy} are a popular non-parametric machine learning model for
function approximation.
The objective is to find a function $F: \mathbb{R}^d \to \mathbb{R}$ that minimizes
\begin{align}
    L(F) = \EE_{\bs{x}, y} \left[ l(y, F(\bs{x})) \right],
\end{align}
where $l: \mathbb{R} \times \mathbb{R} \to \mathbb{R}$ is a loss function, and the expectation is taken over the joint distribution
of the input $\bs{x}$ and the target $y$.
It does this by imposing the requirement that $F$ is as a scaled sum of $M$ decision
trees $f_m: \mathbb{R}^d \to \mathbb{R}$, i.e.
\begin{align}
    F(\bs{x}) = \sum_{m = 1}^{M} \epsilon f_m(\bs{x}), \quad \epsilon \in (0, 1).
\end{align}
where $\epsilon$ is a learning rate or shrinkage parameter.
In the most basic form of the algorithm each tree is constructed to approximate gradient descent on the loss function $L(F)$.
In particular, if we let $F_i = \sum_{m = 1}^{i} f_m$ denote the function after $i$ iterations and
then the $i$-th tree is constructed to approximately minimize the squared error
\begin{align}
    f_i = \text{argmin}_{f} \EE_{\bs{x}, y} \left( f(\bs{x}) - \left. \frac{\partial l(y, \hat{y})}{\partial \hat{y}}\right|_{\hat{y} = F_i(\bs{x})} \right)^2.
\end{align}
using empirical risk minimization and a greedy algorithm to construct the tree.

Various modifications to the basic algorithm have been proposed and implemented such as
regularization, special ways of optimizing the tree, support for categorical functions,
higher order optimization\cite{chen2016xgboost, ke2018lightgbm,prokhorenkova2019catboost}.
In this paper we focus on the implementation of GBTs in the LightGBM library \cite{ke2018lightgbm}
with the understanding that the same principles apply to any other GBT implementation.

\section{Treeffuser/Treeffusion Models}
\label{sec:treefusser}
For $\bs{x} \in \mathbb{R}^d$, $\bs{y} \in \mathbb{R}^m$ the objective of probabilistic
predictions is to produce an estimate of the full  conditional distribution $\Prob[\bs{y}|\bs{x}]$.
This objective is different than in standard regression where the goal
is usually to predict $\EE[\bs{y}|\bs{x}]$.

The most common approach to solve this problem is via parametric models.
This procedure assumes that the distribution $\Prob[\bs{y}| \bs{x}]$
can be well approximated by a parametric family of distributions
\[ \Prob[\bs{y}| \bs{x}] = p[\bs{y}| \theta{(\bs{x})}], \]
where $p$ is a well known distribution (e.g Gaussian) and $\theta(\bs{x})$ is a function
that maps $\bs{x}$ to the parameters of the distribution $p$ (e.g. the mean and covariance of a Gaussian).
Optimization is then performed by finding the function $\theta(\bs{x})$ that minimizes
a proper-scoring rule such as the negative log-likelihood.



\section*{References}


References follow the acknowledgments in the camera-ready paper. Use unnumbered first-level heading for
the references. Any choice of citation style is acceptable as long as you are
consistent. It is permissible to reduce the font size to \verb+small+ (9 point)
when listing the references.
Note that the Reference section does not count towards the page limit.\citet{song2021scorebased}



% Add proper bilbiography file
\bibliographystyle{plain}
\bibliography{neurips_2023}

%%%%%%%%%%%%%%%%%%%%%%%%%%%%%%%%%%%%%%%%%%%%%%%%%%%%%%%%%%%%


\end{document}
